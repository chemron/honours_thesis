\documentclass[a4paper,10pt]{report}
\usepackage{amsmath}
\usepackage{bm}
\usepackage{geometry}
\usepackage{array}
\usepackage{physics}
\usepackage{amssymb}
\usepackage{textcomp}
\usepackage[T1]{fontenc}
\usepackage{gensymb}
\usepackage{tikz}
\usetikzlibrary{shapes, shadows, positioning}
\usepackage[detect-all]{siunitx}
\usepackage{breqn}
\usepackage{subfigure}
\usepackage{geometry}
\usepackage[nottoc,numbib]{tocbibind}
\usepackage{listings}
\usepackage{caption}
\usepackage[utf8]{inputenc}
\usepackage{hyperref}
\usepackage{titlesec}
\usepackage{natbib}

\geometry{a4paper, total={160mm,247mm}, left=25mm, top=25mm,} \graphicspath{
 {figures/} }
\renewcommand{\listfigurename}{Figures}


\captionsetup{justification   = raggedright, singlelinecheck = false}

\hypersetup{colorlinks, citecolor=black, filecolor=black, linkcolor=black,
    urlcolor=black}

\usepackage{titlesec}
\renewcommand{\bibname}{References}


\bibliographystyle{unsrtnat}
\setcitestyle{authoryear,open={(},close={)}}


\titleformat{\chapter}[hang]
  {\normalfont\huge\bfseries}{\thechapter \ }{1em}{}


%opening
\title{Thesis}
\author{Cameron Smith\\
Student ID: 28792912\\
Supervisors: Andrew Casey, Alina Donea}
\date{\today}

\begin{document}

\maketitle

\chapter*{Abstract}



\chapter*{Acknowledgements}

\tableofcontents

\chapter{getting the stereo data}
- figure showing positions and tragectory of stereo, sun earth, SDO and
imaginary satelite for accoustic maps\\
- figure (maybe the same one) showing plot of stereo position in \\
- position data from http://www.srl.caltech.edu/STEREO/docs/position.html \\
-  This is in Heliocentric Earth equatorial (HEEQ): This system has its Z axis
parallel to the Sun's rotation axis (positive to the North) and its X axis
towards the intersection of the solar equator and the solar central meridian as
seen from the Earth. This system is sometimes known as heliocentric solar (HS)\\
- we want active regions to be at the same point of the image.\\
 - to do this, we have to compare stereo and accoustic maps taken at different
 times. \\
 - the sun has a period of T=27.2753 days according in carrington coordinates \\
 - angle between spacecraft and solar farside is given by: \\

\begin{align}
  \theta = \arctan{\frac{y}{x}}
\end{align}

- this gives:
\begin{align}
  t_{stereo} = t_{farside} + \frac{\theta}{2\pi}T
  \intertext{or equivalently}:
  t_{farside} = t_{stereo} - \frac{\theta}{2\pi}T
\end{align}
- using this, for each point in stereo time, i calculated the equivalent farside
time \\
- from this I download the FITS STEREO images that corresponded to the farside
images \\
 - (farside are created every 12 hours between 2010 and now)

 \end{document}

