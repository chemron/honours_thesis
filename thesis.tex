\documentclass[a4paper,10pt]{report}
\usepackage{amsmath}
\usepackage{mathtools}
\DeclarePairedDelimiter{\abs}{\lvert}{\rvert}
\usepackage{bm}
\usepackage{geometry}
\usepackage{array}
\usepackage{physics}
\usepackage{amssymb}
\usepackage{textcomp}
\usepackage[T1]{fontenc}
\usepackage{gensymb}

\usepackage{tikz}
\usetikzlibrary{shapes.geometric, arrows, calc}
\usepackage[nottoc, numbib]{tocbibind}
\tikzstyle{cool} = [rectangle, rounded corners, minimum width=3cm, minimum
height=1cm, text centered, draw=black, fill=gray!30, text width=3cm]
\tikzstyle{arrow} = [thick, ->, >=stealth]
\tikzstyle{line} = [thick, -, >=stealth]

\usepackage[detect-all]{siunitx}
\usepackage{breqn}
\usepackage{subfigure}
\usepackage{geometry}
\usepackage{listings}
\usepackage{caption}
\usepackage[utf8]{inputenc}
\usepackage{hyperref}
\usepackage{titlesec}
\usepackage{natbib}
\usepackage{upgreek}
\usepackage{aas_macros}
\usepackage{doi}
\usepackage{siunitx}
\usepackage{textgreek}


\geometry{a4paper, total={160mm,247mm}, left=25mm, top=25mm,}
\graphicspath{
 {figures/} }
\renewcommand{\listfigurename}{Figures}


\captionsetup{justification   = raggedright, singlelinecheck = false}

\hypersetup{colorlinks, citecolor=black, filecolor=black, linkcolor=black,
    urlcolor=black}

\renewcommand{\bibname}{References}

\bibpunct{(}{)}{;}{a}{}{,}

\hypersetup{
    colorlinks,
    citecolor=black,
    filecolor=black,
    linkcolor=black,
    urlcolor=black
}


\newcommand*\chem[1]{\ensuremath{\mathrm{#1}}}

\newcommand{\threevdots}{%
  \vbox{\baselineskip1ex\lineskiplimit0pt%
  \hbox{.}\hbox{.}\hbox{.}}}



% diagnose: Label(s) may have changed. Rerun to get cross-references right.
% \def\@testdef #1#2#3{%
%   \def\reserved@a{#3}\expandafter \ifx \csname #1@#2\endcsname
%  \reserved@a  \else
% \typeout{^^Jlabel #2 changed:^^J%
% \meaning\reserved@a^^J%
% \expandafter\meaning\csname #1@#2\endcsname^^J}%
% \@tempswatrue \fi}


%opening
\title{Thesis}
\author{Cameron Smith\\
Student ID: 28792912\\
Supervisors: Andrew Casey, Alina Donea}
\date{\today}



\begin{document}

\maketitle

\chapter*{Abstract}



\chapter*{Acknowledgements}

\tableofcontents

\chapter{just getting things down quickly}

\section{getting the stereo data}
- figure showing positions and tragectory of stereo, sun earth, SDO and
imaginary satelite for accoustic maps\\
- figure (maybe the same one) showing plot of stereo position in \\
- position data from http://www.srl.caltech.edu/STEREO/docs/position.html \\
-  This is in Heliocentric Earth equatorial (HEEQ): This system has its Z axis
parallel to the Sun's rotation axis (positive to the North) and its X axis
towards the intersection of the solar equator and the solar central meridian as
seen from the Earth. This system is sometimes known as heliocentric solar (HS)\\
- we want active regions to be at the same point of the image.\\
 - to do this, we have to compare stereo and accoustic maps taken at different
 times. \\
 - the sun has a period of T=27.2753 days according in carrington coordinates \\
 - angle between spacecraft and solar farside is given by: \\

\begin{align}
  \theta = \arctan{\frac{y}{x}}
\end{align}

- this gives:
\begin{align}
  t_{stereo} = t_{farside} + \frac{\theta}{2\pi}T
  \intertext{or equivalently:}
  t_{farside} = t_{stereo} - \frac{\theta}{2\pi}T
\end{align}
- using this, for each point in stereo time, i calculated the equivalent farside
time \\
- from this I download the FITS STEREO images that corresponded to the farside
images \\
 - (farside are created every 12 hours between 2010 and now)



\section{Data pre-processing}
- download the raw fits files of AIA and HMI from jsoc \\
- The FITS meta-data allows us to translate between pixel space and
helioprojective coordinate space.
- using this information, we can rotate the image such that the north pole is at
the highest point of the image, and then crop the image such that the edges of
the sun are at the edge of the image (see figure % From get_aia_png.ipynb
) \\

 - However the 304\AA \ channel of SDO AIA is degrading over time
 \cite{boerner_photometric_2014}, effectively reducing the exposure of the
 image. Figures \ref{fig:aia_degradation} show a comparison in the exposures of
 images taken in 2011, 2015 and 2019 respectively. To account for this
 degradation, the image was clipped to the 95th percentile of the pixel
 intensity and normalised. 


 \begin{figure}[t]%
  \centering
  \subfigure[]{%
    \label{fig:aia_2011}%
    \includegraphics[height=1.5in]{AIA_2011.01.01_00:00:00.png}% 
  }%
  \qquad
  \subfigure[]{%
    \includegraphics[height=1.5in]{AIA_2015.01.01_00:00:00.png}%
    \label{fig:aia_2015}%
  }%
  \qquad
  \subfigure[]{%
    \includegraphics[height=1.5in]{AIA_2019.01.01_00:00:00.png}%
    \label{fig:aia_2019}%
  }%
  \caption[]{
    Images taken by SDO AIA 304\AA \ on the first of January in
    2011 \subref{fig:aia_2011}, 2012 \subref{fig:aia_2015} and 2019
    \subref{fig:aia_2019}. Due to the degradation of the instrument, the
    exposure reduces over time. \textit{Images courtesy of NASA.}}
  \label{fig:aia_degradation}
\end{figure}


\section{STEREO Magnetogram GAN}
- for the gan the input magnetograms and the seismic maps images needed to have
the same dimensions
- while the seismic maps were 180 by 180 pixels, the magnetograms were 1024 by
1024 pixels.

% currently thinking of upsampling seismic maps then downsampling magnetograms,
% but commented out paragraph shows my initial idea

% - to make them the same dimension, the magnetograms were resampled using bicubic
% interpolation to the size of the seismic maps. This interpolation method was
% chosen as it results in a smoother resampling with fewer interpolation artifacts
% \citep{keys_cubic_1981}.


- to make them the same dimension, the seismic maps were resampled using bicubic
interpolation to the size of the magnetograms. This interpolation method was
chosen as it results in a smoother resampling with fewer interpolation artifacts
\citep{keys_cubic_1981}.

\bibliographystyle{mnras.bst}

\bibliography{Bibliography}

 \end{document}

